\let\negmedspace\undefined
\let\negthickspace\undefined
\documentclass[journal]{IEEEtran}
\usepackage[a5paper, margin=10mm, onecolumn]{geometry}
%\usepackage{lmodern} % Ensure lmodern is loaded for pdflatex
\usepackage{tfrupee} % Include tfrupee package

\setlength{\headheight}{1cm} % Set the height of the header box
\setlength{\headsep}{0mm}     % Set the distance between the header box and the top of the text

\usepackage{gvv-book}
\usepackage{gvv}
\usepackage{cite}
\usepackage{amsmath,amssymb,amsfonts,amsthm}
\usepackage{algorithmic}
\usepackage{graphicx}
\usepackage{textcomp}
\usepackage{xcolor}
\usepackage{txfonts}
\usepackage{listings}
\usepackage{enumitem}
\usepackage{mathtools}
\usepackage{gensymb}
\usepackage{comment}
\usepackage[breaklinks=true]{hyperref}
\usepackage{tkz-euclide} 
\usepackage{listings}
% \usepackage{gvv}                                        
\def\inputGnumericTable{}                                 
\usepackage[latin1]{inputenc}                                
\usepackage{color}                                            
\usepackage{array}                                            
\usepackage{longtable}                                       
\usepackage{calc}                                             
\usepackage{multirow}                                         
\usepackage{hhline}                                           
\usepackage{ifthen}                                           
\usepackage{lscape}
\begin{document}

\bibliographystyle{IEEEtran}
\vspace{3cm}


\title{1-1.4-4}
\author{AI24BTECH11026 - Pendem nitesh sri satya$^{*}$% <-this % stops a space
}
\maketitle
\begin{enumerate}
    \item Find the coordinates of the point which divides the line segment joining the points $(4,-3)$ and $(8,5)$ in the ratio $3:1$ internally\\
\solution
Let the position vectors of the points \((4, -3)\) and \((8, 5)\) be represented as \(\vec{A}\) and \(\vec{B}\) respectively. Therefore, we have:
\begin{align}
\vec{A} = 4\vec{i} - 3\vec{j}
\end{align}
\begin{align}
\vec{B} = 8\vec{i} + 5\vec{j}
\end{align}

Let the position vector of the point \(\vec{P}\) that divides the line segment \(\vec{AB}\) in the ratio \(3:1\) internally be \(\vec{P}\).

Using the section formula in vector form, the position vector \(\vec{P}\) is given by:
\begin{align}
\vec{P} = \frac{m\vec{B} + n\vec{A}}{m+n}
\end{align}
where \(m = 3\) and \(n = 1\).

Substituting the values, we get:
\begin{align}
\vec{P} = \frac{3(8\vec{i} + 5\vec{j}) + 1(4\vec{i} - 3\vec{j})}{3+1}
\end{align}
\begin{align}
\vec{P} = \frac{(24\vec{i} + 15\vec{j}) + (4\vec{i} - 3\vec{j})}{4}
\end{align}
\begin{align}
\vec{P} = \frac{(24\vec{i} + 4\vec{i}) + (15\vec{j} - 3\vec{j})}{4}
\end{align}
\begin{align}
\vec{P} = \frac{28\vec{i} + 12\vec{j}}{4}
\end{align}
\begin{align}
\vec{P} = 7\vec{i} + 3\vec{j}
\end{align}

Therefore, the coordinates of the point are \((7, 3)\).
\end{enumerate}
\end{document}

